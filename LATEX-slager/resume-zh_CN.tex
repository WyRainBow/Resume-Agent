% !TEX TS-program = xelatex
% !TEX encoding = UTF-8 Unicode
% !Mode:: "TeX:UTF-8"

\documentclass{resume}
\usepackage{zh_CN-Adobefonts_external} % Simplified Chinese Support using external fonts (./fonts/zh_CN-Adobe/)
%\usepackage{zh_CN-Adobefonts_internal} % Simplified Chinese Support using system fonts
\usepackage{linespacing_fix} % disable extra space before next section
\usepackage{cite}

\begin{document}
\pagenumbering{gobble} % suppress displaying page number

\name{韦宇}

% {E-mail}{mobilephone}{homepage}
% be careful of _ in emaill address
\contactInfo{18933549212}      {3658043236@qq.com}{大模型应用开发/后端开发工程师}
% {E-mail}{mobilephone}
% keep the last empty braces!
%\contactInfo{xxx@yuanbin.me}{(+86) 131-221-87xxx}{}
 
\section{实习经历}
\datedsubsection{\textbf{腾讯云} - 后端开发实习生(Go/Python)}{2025.06 - 2025.10}

\datedsubsection{\textbf{深言科技} - AI应用开发实习生(Go/Python)}{2025.02 - 2025.06}

\datedsubsection{\textbf{美的集团} - 后端开发实习生(Java)}{2024.12 - 2025.01}

\section{项目经验}



\datedsubsection{\textbf{腾讯云域名注册业务}}{}
\begin{itemize}[parsep=0.2ex]
  \item \textbf{搜索服务拆分专项}
    \begin{itemize}[label=\textbf{·},parsep=0.2ex]
      \item 针对域名Check查询挤占连接资源导致核心业务超时问题、主导服务架构拆分
      \item 按读/写属性垂直拆分EPP服务为读写两个独立集群、配置物理隔离连接池
      \item 设计公共备用集群作为容灾方案。增加命令路由切换、根据命令类型分发至对应集群
    \end{itemize}

  \item \textbf{域名黑白名单专项}
    \begin{itemize}[label=\textbf{·},parsep=0.2ex]
      \item 从0到1实现了域名黑白名单模块、提供统一的名单管控
      \item 多级缓存架构:构建SDK本地缓存 + Redis分布式缓存架构、通过Guava Cache和5秒过期策略、支撑近万QPS峰值、本地缓存命中率97\%以上
      \item 数据一致性保障:实现准实时-增量-全量多重保障、通过写DB+发MQ、Binlog监听、定时任务确保数据一致性
    \end{itemize}

  \item \textbf{风险SQL治理专项}
    \begin{itemize}[label=\textbf{·},parsep=0.2ex]
      \item 优化100余条高风险SQL、扫描行数从超百万行降至万行以下、执行时间提升80\%
      \item 通过强制索引、JOIN优化等手段确保查询毫秒级响应
      \item 采用代码解耦、游标分页处理大数据量查询
      \item 推动复杂查询向RO只读实例迁移、降低主库负载
    \end{itemize}
\end{itemize}


\datedsubsection{\textbf{深言科技 - 语鲸 DeepResearch - AI Agent 智能搜索系统}}{}
\begin{itemize}[parsep=0.2ex]
  \item \textbf{项目描述}:基于清华大模型构建的自主式深度研究(DeepResearch)智能体。具备复杂意图识别、多源混合检索、长文本深度阅读与结构化研报生成能力、解决传统搜索信息碎片化痛点。
  \item \textbf{核心职责与产出}:参与AI搜索链路和主导广告投放系统开发。从用户提问→多源检索→长文阅读→报告输出全流程上、我参与了把用户的模糊问题拆解成可执行任务、将各类数据源打通、让模型读懂并输出可用结果。同时建设配套的投放与数据效果反馈
    \begin{itemize}[label=\textbf{·},parsep=0.2ex]
      \item \textbf{Agent 驱动的推理与规划 (Reasoning \& Planning):} 设计基于 LLM 的意图拆解模块、将复杂长查询转化为 \textbf{CoT (思维链)} 任务序列。实现\textbf{基于推理的查询扩展}、结合领域知识库与用户行为日志、利用 LLM 自动生成并校验同义词及行业术语、提升长尾查询的召回率。
      \item \textbf{多源融合检索架构 (Hybrid Search):} 基于 \textbf{BGE-M3} 实现“稠密向量+稀疏关键词”的混合检索链路。并行编排 Bing Search API(广度)、垂类教育库/知识图谱(深度)及 Milvus 向量检索。结合 LLM 查询改写与结果排序提升召回与准确性
      \item \textbf{RAG 深度阅读与抗幻觉生成:} 构建上下文感知生成模块、利用 RAG 技术将高置信度片段注入 LLM Context Window。通过 Prompt Engineering 引导模型进行多源信息比对与事实核查、输出Markdown格式的结构化深度研报(包含摘要、提纲及溯源链接)。
      \item \textbf{广告数据闭环链路:} 主导搜索结果页的广告投放系统构建。实现从曝光、点击(CTR)到转化的全链路埋点追踪、基于用户行为数据进行闭环反馈、迭代优化搜索排序算法和内容推荐策略。
    \end{itemize}
\end{itemize}



\section{开源经历}
\datedsubsection{\textbf{Seata-go 社区}}{阿里开源分布式事务框架}
\begin{itemize}[parsep=0.2ex]
  \item 仓库: \textit{https://github.com/apache/incubator-seata-go}
  \item 实现本地缓存计数器PR 记录:\textit{https://github.com/apache/dubbo-admin/pull/1345}
  \item 主要负责社区日常Issue维护与答疑、问题复现与定位
\end{itemize}

\datedsubsection{\textbf{SellSense AI:全域知识增强的大模型销售Agent}}{}
\begin{itemize}[parsep=0.2ex]
  \item \textbf{Python 手撕版:} LlamaIndex、Milvus、BGE、BM25、Search API、ReAct/Reflection/Memory Workflow
  \item \textbf{仓库:} \textit{https://github.com/WyRainBow/My-Agent}(可现场演示)
  \item \textbf{RAG检索与智能工作流:} 基于LlamaIndex+Milvus+BGE搭建本地知识检索、结合BM25混合召回以及Search API获取的实时信息、由多Agent(ReAct/Reflection/Memory)协作完成查询拆解、推理与结果校验。
  \item \textbf{成果:} 在自建技术社区与开源平台持续分享实践经验、帮助后端与大模型同学快速入门并迭代方案、项目累计获得 40+ Star(增长中)
\end{itemize}

\section{专业技能}
\begin{itemize}[parsep=0.2ex]
  \item \textbf{后端:} 熟悉Java编程语言、Golang编程语言等原理
  \item \textbf{数据库:} 熟悉 MySQL、MongoDB、ES、Milvus 等主流数据库原理。有非常优秀的 SQL 调优经验
  \item \textbf{Redis:} 熟悉Redis底层数据结构、分布式锁等机制。熟悉缓存击穿、穿透、雪崩概念
  \item \textbf{计算机网络:} 熟悉TCP、UDP、HTTP、HTTPS等网络协议。掌握TCP三次握手、四次挥手等机制
  \item \textbf{操作系统:} 熟悉进程、线程、虚拟内存、I/O多路复用等。掌握进程间通信和多线程同步技术
  \item \textbf{AI:} 了解AI Agent、RAG、FunctionCall、LLM(如阿里百炼、DeepSeek)Prompt提示词工程
\end{itemize}

\section{教育经历}
\datedsubsection{\textbf{广东药科大学} - 计算机科学与技术 -  \textit{本科}}{2022.09 - 2026.06}
\ \textbf{荣誉:} 全国大学生数学建模省一等奖、人工智能和大数据比赛省二等奖

%% Reference
%\newpage
%\bibliographystyle{IEEETran}
%\bibliography{mycite}
\end{document}

