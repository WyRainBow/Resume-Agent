% !TEX TS-program = xelatex
% !TEX encoding = UTF-8 Unicode
% !Mode:: "TeX:UTF-8"

\documentclass{resume}
\usepackage{zh_CN-Adobefonts_external}
\usepackage{linespacing_fix}
\usepackage{cite}

\begin{document}
\pagenumbering{gobble}

\name{某某}

\contactInfo{000-0000-0000}{placeholder@mail.com}{某方向工程师}

\section{实习经历}
\datedsubsection{\textbf{实习经历一} - 某职位}{2025.06 - 2025.10}

\datedsubsection{\textbf{实习经历二} - 某职位}{2025.02 - 2025.06}

\datedsubsection{\textbf{实习经历三} - 某职位}{2024.12 - 2025.01}

\section{项目经验}

\datedsubsection{\textbf{项目一}}{}
\begin{itemize}[parsep=0.2ex]
  \item \textbf{子项目甲}
    \begin{itemize}[label=\textbf{·},parsep=0.2ex]
      \item 描述该子项目的主要目标和解决的问题
      \item 概述采用的核心技术手段或架构思路
      \item 说明实现过程中的关键策略或容灾措施
    \end{itemize}

  \item \textbf{子项目乙}
    \begin{itemize}[label=\textbf{·},parsep=0.2ex]
      \item 介绍从0到1搭建某模块的背景与价值
      \item 说明缓存或性能优化的思路与结果
      \item 描述数据一致性或稳定性保障方案
    \end{itemize}

  \item \textbf{子项目丙}
    \begin{itemize}[label=\textbf{·},parsep=0.2ex]
      \item 总结优化高风险操作的范围与收益
      \item 概括查询调优、索引策略等具体动作
      \item 解释资源隔离或负载转移方式
    \end{itemize}
\end{itemize}

\datedsubsection{\textbf{项目二}}{}
\begin{itemize}[parsep=0.2ex]
  \item \textbf{项目描述}:概述一个具备多模态检索、长文阅读与结构化输出能力的智能系统,强调其解决的痛点与特性。
  \item \textbf{核心职责与产出}:描述在需求拆解、链路打通以及配套平台建设中的角色与贡献。
    \begin{itemize}[label=\textbf{·},parsep=0.2ex]
      \item \textbf{模块一}: 说明如何利用大模型进行推理规划与查询扩展,提升召回能力。
      \item \textbf{模块二}: 概括多源融合检索架构,指出使用的检索方式与调度策略。
      \item \textbf{模块三}: 描述RAG或抗幻觉生成的实现思路、Prompt策略与输出形式。
      \item \textbf{模块四}: 介绍广告或数据闭环链路的建设,涵盖埋点、分析与反馈机制。
    \end{itemize}
\end{itemize}

\section{开源经历}
\datedsubsection{\textbf{社区贡献一}}{某分布式项目}
\begin{itemize}[parsep=0.2ex]
  \item 仓库: \textit{https://example.com/repo1}
  \item 简述提交的核心PR或Issue处理经验
  \item 说明在社区内承担的协作职责
\end{itemize}

\datedsubsection{\textbf{社区贡献二}}{}
\begin{itemize}[parsep=0.2ex]
  \item \textbf{组件一:} 列举涉及的技术栈与能力范围
  \item 仓库: \textit{https://example.com/repo2}{可演示}
  \item \textbf{能力二:} 描述检索、知识构建或多Agent流程的实现
  \item \textbf{成果:} 简述分享传播与社区反馈
\end{itemize}

\section{专业技能}
\begin{itemize}[parsep=0.2ex]
  \item \textbf{后端:} 熟悉若干编程语言或服务框架
  \item \textbf{数据库:} 了解常见数据库及调优思路
  \item \textbf{缓存:} 掌握缓存策略与典型问题处理
  \item \textbf{网络:} 熟悉常见网络协议与连接管理
  \item \textbf{操作系统:} 理解进程线程与资源管理机制
  \item \textbf{AI:} 了解Agent、RAG、FunctionCall与Prompt工程
\end{itemize}

\section{教育经历}
\datedsubsection{\textbf{某高校} - 某专业 -  \textit{本科}}{2022.09 - 2026.06}
\ \textbf{荣誉:} 例如学科竞赛、省级奖项等

\end{document}


